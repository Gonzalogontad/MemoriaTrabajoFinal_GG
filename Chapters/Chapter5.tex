% Chapter Template

\chapter{Conclusiones} % Main chapter title

\label{Chapter5} % Change X to a consecutive number; for referencing this chapter elsewhere, use \ref{ChapterX}


%----------------------------------------------------------------------------------------

%----------------------------------------------------------------------------------------
%	SECTION 1
%----------------------------------------------------------------------------------------

\section{Conclusiones generales }

El presente trabajo surge de la necesidad de la empresa Posthac S.A. de automatizar un proceso de control de productos. Para ello, se diseñó e implementó un prototipo de banco de pruebas siguiendo un listado de requerimientos, los cuales se lograron cumplir en su totalidad.

En el desarrollo de este prototipo se aplicaron muchos de los conocimientos adquiridos a lo largo de la carrera de especialización. Entre los conocimientos aplicados, se pueden destacar los adquiridos en las siguientes asignaturas:

\begin{itemize}
\item Gestión de proyectos: seguir un plan de proyecto desde el comienzo, hizo posible identificar desvíos y tomar medidas correctivas logrando así cumplir con el alcance del proyecto a pesar de las demoras ocurridas.
\item Ingeniería de software: se aplicaron conceptos de modularización y reutilización de código, uso de patrones de diseño y control de versiones.
\item Protocolos de comunicación en sistemas embebidos: se utilizaron protocolos de comunicación serie y de redes inalámbricas WIFI, afianzando así los conocimientos sobre la temática.
\item Sistemas operativos de tiempo real: la parte más importante del software se desarrolló sobre un RTOS, y se aplicaron conceptos de sincronización y comunicación de tareas y uso compartido de recursos.
\item Diseño de Circuitos Impresos: Se diseñaron y fabricaron circuitos impresos utilizando las técnicas y el software enseñados en la asignatura.
\end{itemize}

En cuanto al cumplimiento de los objetivos, el prototipo permite la reducción de tiempos que se buscaba, ya que puede realizar hasta seis pruebas a la vez requiriendo una mínima o nula intervención del operario durante el proceso. De este modo se desliga al operario de la responsabilidad de efectuar mediciones y así poder realizar otras tareas en simultaneo, como la desconexión de equipos ya probados y conexión de equipos a probar. Además, el echo de que el operario no necesite hacer mediciones supone una mejora de la calidad del proceso debido a que reduce la posibilidad del error humano en las mediciones.




%----------------------------------------------------------------------------------------
%	SECTION 2
%----------------------------------------------------------------------------------------
\section{Próximos pasos}

Con el objetivo de mejorar la calidad de los controles y la trazabilidad de fallas de todas las líneas de productos que se fabrican en la empresa, se pueden identificar las siguientes oportunidades de mejora y ampliación:

\begin{itemize}
\item Desarrollo de un módulo adicional y software para prueba de sensores de movimiento infrarrojo pasivos.
\item Desarrollo de un módulo adicional para prueba de entrada de drivers de LEDs con mediciones de potencia, factor de potencia y distorsión armónica.
\item Integración del banco de pruebas con una base de datos para recolectar resultados a fin de obtener datos estadísticos en forma automática.
\end{itemize}
